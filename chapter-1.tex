\chapter{Introduction}
\label{INTRO}
Teaching programming has become increasingly important over the last years. Moreover, teaching children how to program at school is at least equally important as teaching them at university. As a result, programming languages and applications need to be devised for an educational background. The following chapter gives an introduction to this subject and will explain why and how a domain-specific language can help teach programming. The methodology elaborates how to design a programming language and how to integrate this approach into a holistic solution. At the end of this section, the structure of this research is outlined.

\section{General}
\label{INTRO_GEN}
In Schnabel's research 2011 he said:
\begin{quote}
"only through giving students deep computer science (CS) knowledge can we expect to have a new generation that can create—not just consume—the next wave of computing innovations."\cite{Schnabel2011}
\end{quote}
However, to create this new generation, it is necessary to find new approaches to teach programming. One method is to combine gamification with programming or use hardware with programming. Section \ref{LIT_PROJ} shows related projects for each approach. Soloway et al. suggest to pack expert knowledge into the natural language.\cite{Soloway1982} Domain-specific languages (DSLs) offers this opportunity by creating a language style, which has a natural feel. However, teaching programming does not depend solely on the language; just as significant are the findings of the written program.\cite{Kahn1995} As mentioned above, one approach is to add other stimuli or charms. Adding music or sound to the learning procedure could increase the learning curve—for novices as well as children.

\section{Precedence of DSLs and Sound Generation}
\label{INTRO_PRECE}
DSLs are not a new technology and can be traced back to Ross, 1978.\cite{Ross1978} Usually, DSLs are a common way to offer developers a simple mechanism to interwork with higher complex software or give non-developers access to software projects to manipulate them. Many studies discuss the advantages of DSLs. Hofer et al. formulated that "simple interpreters can be quickly derived and implemented" and "it is easy to extend the DSL".\cite{Hofer2011} Fowler also names two reasons why developers should be interested in DSLs; they "improve programmer’s productivity" and they make program parts "easier to understand".\cite[p. xxi]{Fowler2010} This circumstance makes it evident that DSLs are an excellent approach to teach novices programming, regardless, the most notable disadvantage of DSLs are their complexity in development.\cite{Mernik2005}

The combination of programming and sound generation is not a new approach. The project \textit{Sonic Pi}\footnote{Sonic Pi - \url{http://sonic-pi.net}} provides a solution which focuses on sound generation through text description and offers an interactive environment.\cite{Aaron2013} The system is based on physical-modelling-synthesis, which means that user should have a minimum knowledge of classical sound synthesising techniques like wave amplitudes. Therefore, this approach adds a further knowledge-domain to the programming and music-domain. A more accessible approach is desirable.

\section{Combining Software and Music}
\label{INTRO_SOFT}
The purpose of this study is to describe and examine an approach to design a programming language for programming novices which is simple to read, understand and learn. The objective of the language is to teach the primary and elementary programming concepts like variable declaration, expressions, arithmetic or method invocation. This means that the language offers a suitable balance between the proximity to natural language and real computer science languages. Scala is used as global-purpose language (GPL) to develop the DSL, especially the parser and the corresponding interpreter. This work concentrates on the implementation of external DSLs, but discusses these in relation to internal DSLs and compares them. The DSL itself is not a Turing complete language by design and serves only one purpose: to create music through programming notes. The sound, or, more precisely, the music, which is a result of the corresponding DSL, is generated by a application written in Scala.

To provide a holistic approach, the user-interface is implemented as a web-based editor. The intention is to write and explore the DSL in a simple to use web-browser application, with the focus to provide fast feedback like syntax highlighting, as well as error feedback and the possibility to playback the written music. This web-application is also developed with Scala. The individual web components such as the editor, the music playback and the user interaction are written with the assistance of Scala.js.

\section{Objectives and Methodology}
\label{INTRO_METHOD}
This thesis is organised as follows: Chapter \ref{THEO} describes and provides the necessary theoretical knowledge and the primary intention. It describes DSLs in general, compares internal with external DSLs and shows how Scala acts as the center point of the project. Furthermore, MIDI as sound-medium is explained. Scala.js is identified as the central component for the web-based user-interface. Afterwards, chapter \ref{LIT} presents ongoing research projects and existing research resources in order to establish the state-of-the-art in the area of teaching programming, DSL development and sound generation by code. The methodology in chapter \ref{IMPL} firstly explains the technical implementation and illustrates the architecture. Secondly, the implementation is broken down into two parts. The first part discusses the DSL concept and implementation through Scala parser combinators in particular. The second part comprises the web-application, which serves as the client-server model for the holistic application. Beyond that, the client and the server make use of additional libraries such as multiple Scala.js façades, which are built as independent projects. In chapter \ref{RESULTS} the entire application is viewed at large. The inclusion of various DSL script examples and a minor case study serve as a basis for analysis. Finally, chapter \ref{DISCUSSION} provides a brief discussion of the results and gives a conclusion.



