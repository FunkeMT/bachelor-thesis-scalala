\chapter{Conclusion}
\label{CONCLUSION}
A general introduction to this thesis, the topic and the research objectives were given in chapter \ref{INTRO}. To provide the reader with the underlying theoretical principles, chapter \ref{THEO} discussed the difference between internal and external DSLs and defined Scala as backbone GPL. Related literature was reviewed in chapter \ref{LIT} to provide a necessary background of the thesis's research field and to point out a gap in the current research. Besides, the chapter presented some related work and discussed their advantages as well as disadvantages in correlation to this thesis. Chapter \ref{ARCH} defined and explained the applications architecture. The defined requirements in \ref{INTRO_SOFT} were implemented in detail in chapter \ref{IMPL}. The results were extensively discussed in chapter \ref{RESULTS}, and a minor case study was given which provided useful results related to the DSL's properties. The discussion in chapter \ref{DISCUSSION} compared the developed DSL to Turing completeness, stated the advantages of the ScalalaKata web-application and raised questions for future work.

The thesis successfully implemented a holistic web-application, integrating the developed language Scalala. Therefore, the initial aim of this thesis was successfully achieved. An external DSL was developed for novices which can be used in an educational context. Fluency and proximity to plain English were achieved through consulting related studies and research in the field of teaching novices and children.

The web-application presents a cross-platform as well as a mobile approach to provide an accessible application for everyone. Through publishing the developed libraries to a software distribution service, this thesis has made a contribution to the open source community and enables further research.

The minor case study showed that the language is accessible to both domains and illustrated correlations to significant case studies from other research and stated the importance of comprehensive studies. The interviews with the music domain experts showed, that the developed language does not reach the full cardinality of the music domain and some language implementations do not fulfil all music domain requirements. This observation outlined the importance of the communication process between developers and domain experts in implementing a new language, which is stated by Fowler.\cite{Fowler2010}

Wing declared "Computational Thinking" a vital skill for novices and children.\cite{Wing2006} This could be applied to Schnabel's statement to provide the next-generation with fundamental knowledge.\cite{Schnabel2011} However, this thesis showed that the effort to create a new language from scratch and provide an accessible interface should not be neglected. Nevertheless, to create solutions which enhance "Computational Thinking" is worth the effort.