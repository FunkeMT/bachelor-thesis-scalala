\chapter{Appendix - Minor Case Study User Behaviour}
\label{APPENDIX_B}

\begin{longtable}{p{50pt}|p{320pt}}
\rowcolor{htwg-teal} 
\textbf{User}		& \textbf{Behaviour}     \\
\endhead
(A)                  		& \begin{itemize}
\item  Executed the initial example script before analysing the script.
\item  Modified the script through deleting 'musician' variables of the 'play' loop, and not the variable definition.
\item  Tried to comment out 'musician' variables instead of deleting it. The subject added various comment symbols in front of the lines, like '//', which all ended in an error.
\item  Instead of trial-and-error, the subject searched for a simple sheet-note representation of a popular music piece. The subject adopted the notes into the DSL script syntax, by comparing the example and the note-sheets. The result was not the expected music piece.
\item  The subject asked (the author) for further syntax features and tried through 'Ctrl + Space' to get additional syntax; without effect.
\end{itemize} \\
(B)                 		& \begin{itemize}
\item  Analysed the initial example script before executing it.
\item  Modified the script by adding one piano note.
\item  Deleted all musician variable definitions 'and' its usages.
\item  Modified the piano variable and wrote a well-known children's song from memory.
\item  The subject asked (the author) for further music elements. The subject did not recognise the octaves from the initial example.
\end{itemize} \\
(C)                 		& \begin{itemize}
\item Executed the initial example script before analysing the script. (identically to user A)
\item Modified the script through deleting 'musician' variables of the 'play' loop, and not the variable definition. (identically to user A)
\item The subject adjusted the tempo and compared it to the music notation. The subject mentioned that it is obvious that the tempo unit is 'bpm'.
\item The subject typed a melody, played it, modified it, played it, and so on.
\item The subject asked (the author) for further music notation features.
\item The user did not produce an error.
\end{itemize} 
\end{longtable}
