\chapter*{Abstract}
\setheader{Abstract}

Internal Domain-Specific Languages (DSLs) are typically well known by engineers and developers in form of APIs. External DSLs are used by business employees–the domain experts–to modify enterprise software without having to rely on the developers. However, both are experts in their own language.
The aim of this study is to create a music-notation language for both, technical and non-technical users. Through the usage of the DSL, basic concepts of programming languages as well as simple music notation can be taught.
Already conducted research has shown how to create music with written language or how sheet music can be created programmatically, but both of them are limited to people with an understanding of physical-modelling-synthesis or knowledge of programming.
This thesis investigates the capabilities to write music notation, or rather, to program music without programming expertise.
It used the cardinality of external DSLs in Scala to design a user-friendly language.
Beyond that, the generation and the playback of sound and written music, as well as the whole pipeline between the language interpreter and the web-based interface with Scala.js, was implemented.
As a result, the thesis presents a holistic web application to explore the capabilities of the developed language and discuss the application of the language to teach programming novices.
